\documentclass[11pt, oneside]{article}   	% use "amsart" instead of "article" for AMSLaTeX format
\usepackage{geometry}                		% See geometry.pdf to learn the layout options. There are lots.
\geometry{letterpaper}                   		% ... or a4paper or a5paper or ... 
%\geometry{landscape}                		% Activate for rotated page geometry
%\usepackage[parfill]{parskip}    		% Activate to begin paragraphs with an empty line rather than an indent
\usepackage{graphicx}				% Use pdf, png, jpg, or eps§ with pdflatex; use eps in DVI mode
\usepackage{amsmath}								% TeX will automatically convert eps --> pdf in pdflatex		
\usepackage{amssymb,amsthm}
\usepackage[utf8]{inputenc}

%SetFonts

%SetFonts


\title{Nonnegative matrix factorization for interactive topic modeling and document clustering}
\author{Sarah Lu}
%\date{}							% Activate to display a given date or no date

\begin{document}
\maketitle
\section{Experimental Results}
%\subsection{}
\begin{Large}
In section 5, authors  present the empirical evidences that support NMF as a successful document clustering and topic modeling method. They compare the clustering quality between K-means and NMF; Within the NMF algorithms, they compare the multiplicative updating (MU) algorithm and the alternating nonnegative least squares (ANLS) algorithm in terms of their clustering quality and convergence behavior, as well as sparseness and consistency in the solution.  

The data are described as follows:
\begin{center}
 \begin{tabular}{||c c c c||} 
 \hline
 Data set & Terms & Documents & Ground-truth clusters \\ [0.5ex] 
 \hline\hline
 TDT2 & 26,618 & 8,741 & 20 \\ 
 \hline
 Reuters & 12,998 & 8,095 & 20 \\
 \hline
 20 Newsgroups & 36,568 & 18,221 & 20 \\
 \hline
 RCV1 & 20,338 & 15,168 & 40 \\
 \hline
 NIPS14-16 & 17,583 & 420 & 9 \\ [1ex] 
 \hline
\end{tabular}
\end{center}

They  process each term-document matrix $A$ in two steps. First, they normalize each column of $A$ to have a unit $L2$-norm. Conceptually, this makes all the documents have equal lengths. Next, they compute the normalized-cut weighted version of $A$:
\begin{center}
$D = diag(A^{T}A1_n), A\leftarrow AD^{-1/2}$
\end{center}

For K-means clustering, they used the standard K-means with Euclidean distances, through a batch-update phase and a more time-consuming online-update phase in Matlab.

For the ANLS algorithm for NMF, they used the block principal pivoting algorithm to solve the NLS subproblems and the stopping criterion was $\varepsilon = 10^{-4}$. For the MU algorithm for NMF, they used another stopping criterion:
\begin{center}
$\|H^{i-1}-H^{i}\|_{F}/\|H^{i}\|_{F}\leq\varepsilon$
\end{center}

\textbf Clustering quality: Normalized Mutual information (NMI) is calculated by:
\begin{center}
NMI $= \frac{I(C_{ground_truth},C_{computed})}{[H(C_{ground_truth})+H(C_{computed})]/2}=\frac{\sum_{h,l} n_{h,l}\log \frac{n\cdotn_{h,l}}{n_{h}n_{l}}}{(\sum_{h} n_{h}\log \frac{n_{h}}{n}+\sum_{l} n_{l}\log \frac{n_{l}}{n})/2}$
\end{center}

\section{UTOPIAN: User-driven Topic Modeling via Interactive NMF}

In this section, the authors present a visual analytics system called UTOPIAN (User-driven Topic Modeling Based on Interactive NMF). UTOPIAN provides a visual overview of the NMF topic modeling result. Beyond the visual exploration of the topic modeling result in a passive manner, UTOPIAN provides various interaction capabilities that can actively incorporate user inputs to topic modeling processes.\\\\
Topic keyword refinement: This interaction allows users to change the weights corresponding to keywords so that the meaning of the topic can be refined.\\
Topic merging: This interaction merges multiple topics into one.\\
Topic splitting: It splits a particular topic into the two topics. To guide this splitting process, users can assign the reference information for the two topics.\\
Document-induced topic creation: This interaction creates a new topic by using user-selected documents as seed documents.\\
Keyword-induced topic creation. It creates a new topic via user-selected keywords. For instance, given the summary of topics as their representative keywords, users might want to explore more detailed (sub-)topics about particular keywords.

\section{Conclusions and Future Directions}
In this paper, the authors have presented nonnegative matrix factorization (NMF) for document clustering and topic modeling. They have first introduced the NMF formulation and its applications to clustering. Next, they have presented the flexible algorithmic framework based on block coordinate descent (BCD) as well as its convergence property and stopping criterion. Based on the BCD framework, they discussed two important extensions for clustering, the sparse and the weakly-supervised NMF,and their method to determine the number of clusters. Experimental results on various real-world document data sets show the advantage of their NMF algorithm in terms of clustering quality, convergence behavior, sparseness, and consistency. Finally, they presented a visual analytics system called UTOPIAN for interactive visual clustering and topic modeling and demonstrated its interaction capabilities such as topic splitting/merging as well as keyword-/document-induced topic creation.



















\end{Large}

\end{document}  